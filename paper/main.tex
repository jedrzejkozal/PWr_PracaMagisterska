\documentclass[a4paper, 10 pt, conference]{ieeeconf}
\overrideIEEEmargins
\usepackage{polski}
\usepackage{amsmath}
\usepackage{graphicx}
\usepackage[utf8]{inputenc}
\usepackage[T1]{fontenc}
\usepackage{textcomp}
\usepackage[english]{babel}
\usepackage{hyperref}

\newcommand{\bb}{\textbf}

% Listingi
\usepackage{listings}
\usepackage{xcolor}
\lstdefinestyle{mystyle}{
	backgroundcolor=\color{gray!5!white},
	commentstyle=\color{green!50!black},
	keywordstyle=\color{magenta},
	numberstyle=\tiny\color{black!50!white},
	stringstyle=\color{purple},
	basicstyle=\footnotesize,
	breakatwhitespace=false,
	breaklines=true,
	captionpos=b,
	keepspaces=true,
	numbers=left,
	numbersep=5pt,
	showspaces=false,
	showstringspaces=false,
	showtabs=false,
	tabsize=2
}
\lstset{style=mystyle}

% The following packages can be found on http:\\www.ctan.org
\usepackage{graphics} % for pdf, bitmapped graphics files
\usepackage{amsmath} % assumes amsmath package installed
\usepackage{amssymb}  % assumes amsmath package installed
\usepackage{tikz}

\title{\LARGE \bf
Analysis of the effectiveness of recursive
networks in the classification task
}

\author{\parbox{2 in}{\centering Jędrzej Kozal \\
        Wrocław University of Science and Technology\\
        {\tt\small 218557@student.pwr.edu.pl}}
}


\begin{document}

\maketitle
\thispagestyle{empty}
\pagestyle{empty}

\selectlanguage{english}
\begin{abstract}

Recurrent Neural Networks are class of models designed to process sequences. Most of the typical fields of applications include natural language processing, handwriting recognition and generation, or text, music or images generation. In this work emphasis was put on ReNet architecture designed to solve image classification task. Modification based on Hilbert Curve was introduced to ReNet and obtained accuracy was very close to results obtained for original ReNet network. Modification also provided significant training time reduction. We compared ReNet networks to convolutional networks and found that the latter proved to be superior.

\end{abstract}


\section{INTRODUCTION}

\cite{Goodfellow-et-al-2016}

 

\section{RELATED WORK}

\section{METHODS}

\section{EXPERIMENT SETUP}

\section{RESULTS}

\section{CONCLUSIONS}



\bibliographystyle{unsrt}
\bibliography{refs}

\end{document}
