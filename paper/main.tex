\documentclass[a4paper, 10 pt, conference]{ieeeconf}
\overrideIEEEmargins
\usepackage{polski}
\usepackage{amsmath}
\usepackage{graphicx}
\usepackage[utf8]{inputenc}
\usepackage[T1]{fontenc}
\usepackage{textcomp}
\usepackage[english]{babel}
\usepackage{hyperref}

\newcommand{\bb}{\textbf}

% Listingi
\usepackage{listings}
\usepackage{xcolor}
\lstdefinestyle{mystyle}{
	backgroundcolor=\color{gray!5!white},
	commentstyle=\color{green!50!black},
	keywordstyle=\color{magenta},
	numberstyle=\tiny\color{black!50!white},
	stringstyle=\color{purple},
	basicstyle=\footnotesize,
	breakatwhitespace=false,
	breaklines=true,
	captionpos=b,
	keepspaces=true,
	numbers=left,
	numbersep=5pt,
	showspaces=false,
	showstringspaces=false,
	showtabs=false,
	tabsize=2
}
\lstset{style=mystyle}

% The following packages can be found on http:\\www.ctan.org
\usepackage{graphics} % for pdf, bitmapped graphics files
\usepackage{amsmath} % assumes amsmath package installed
\usepackage{amssymb}  % assumes amsmath package installed
\usepackage{tikz}
\usetikzlibrary{positioning} 
\usepackage{makecell}

\title{\LARGE \bf
Analysis of the effectiveness of recursive
networks in the classification task
}

\author{\parbox{2 in}{\centering Paweł Ksieniewicz \\
        Wrocław University of Science and Technology\\
        {\tt\small pawel.ksieniewicz@pwr.edu.pl}}
        \hspace*{ 0.3 in}
        \parbox{2 in}{\centering Jędrzej Kozal \\
        Wrocław University of Science and Technology\\
        {\tt\small 218557@student.pwr.edu.pl}}
}


\begin{document}

\maketitle
\thispagestyle{empty}
\pagestyle{empty}

\selectlanguage{english}
\begin{abstract}

Recurrent Neural Networks are class of models designed to process sequences. Most of typical fields of applications include natural language processing, recognition of handwriting and generation of text, music or images. In this work emphasis was put on a ReNet architecture designed to solve an image classification task. A modification based on a Hilbert curve was introduced to the ReNet and obtained accuracy was very close to results acquired for the original ReNet network. The modification also provided significant training time reduction for some datasets. Comparison of ReNet networks to convolutional networks proved that the latter are superior.

\end{abstract}


\section{INTRODUCTION}

\cite{Goodfellow-et-al-2016}

 

\section{RELATED WORK}

In \cite{DBLP:journals/corr/VisinKCMCB15} ReNet architecture was introduced. It is alternative to convolutional networks, that enables to learn representation of an image, that can be used for classification purpose. Convolutional network computes activation based on the filters applied locally to part of an image. ReNet by using 4 recurrent neural networks can incorporate information scattered across whole image.

For convenience from now on we will refer to input as an image, but it can be also output of previous layer. We can describe inputs as tensor $X = \{x_{i,j}\}, X \in \mathbb{R}^{w \textrm{x} h \textrm{x} c}$, where $w, h$ are size of an image and $c$ is a number of channels. Image is divided into block of pixels called patches. Every patch is of size: $w_p$, $h_p$, therefore there are $(I \times J),I=\frac{w}{w_p}, J=\frac{h}{h_p}$ patches in the whole image. Set of all patches in image $X$ is defined as $P = \{p_{i,j}\}, P \in \mathbb{R}^{w_p \textrm{x} h_p \textrm{x} c}$. In first part of ReNet algorithm we take each columns of patches and feed it to 2 recurrent neural networks: 

\begin{gather}
	v_{i,j}^{F} = f_{VFWD} (v_{i,j-1}^F, p_{i,j}), \\
    v_{i,j}^{R} = f_{VREV} (v_{i,j+1}^R, p_{i,j}),
\end{gather}

where $f_{VFWD} (v_{i,j-1}^F, p_{i,j})$ can be activation of plain RNN, LSTM cell or GRU cell. Concatenated activation can be described as $V = \{v_{i,j}\}_{i=1,...,I}^{j=1,...,J}$. Tensor $V$ is input of another two recurrent layer networks, swapping rows from left to right, and right to left. Operation is analogues to vertical swap, with $f_{HFWD}, f_{HREV}$ computing activations $H = \{h_{i,j}\}$. Transformation $\Phi$ of each layer can be described with introduced designation as:

\begin{equation}
	\Phi: X \rightarrow V \rightarrow H
\end{equation}

Output of layer can be feed to another ReNet layer, or can flatted and feed to fully connected layer.

\section{METHODS}

\subsection{Hilbert curve}

\begin{figure}
	\centering
	\begin{tikzpicture}
	\draw (0,0) node[below] {1} -- (0,2) node[above] {2} -- (2,2) node[above] {3} -- (2,0) node[below] {4};
	\draw (0,2.5) node[above] {1} -- (0.66,2.5) node[above] {2} -- (1.32,2.5) node[above] {3} -- (2,2.5) node[above] {4};
	\draw (3,0) node[below] {1} -- (3.75,0) node[below] {2} -- (3.75,0.75) -- (3,0.75) -- (3,1.5) -- (3,2.25) -- (3.75,2.25) -- (3.75,1.5) -- (4.5,1.5) -- (4.5,2.25) -- (5.25,2.25) -- (5.25,1.5)  -- (5.25,0.75) -- (4.5,0.75) -- (4.5,0) -- (5.25,0);
	\draw (6.25,0) -- (6.25,0.4);
	\end{tikzpicture}
\caption{First 3 elements of sequence creating Hilbert Curve.}
	\label{fig:hilbert}
\end{figure}

Hilbert Curve $\mathcal{H}$ is space filling curve with fractal structure. It is defined by sequence of curves defined recursively, what be described as $k(n+1) = f(k(n))$. In this case transformation $f$ compounds of duplicating and rotating of $n$-th degree curve. First 3 elements of sequence creating Hilbert Cureve are shown in figure \ref{fig:hilbert}.

We can obtain Hilbert Curve in limit:

\begin{equation}
	\mathcal{H} = \lim_{n \rightarrow + \infty } k(n)
\end{equation}

By doing so, we can fill every point of unit square, hence name of this mathematical object - space filling curve. Hilbert Curve have other interesting quantities. $N$-th curve from sequence defining Hilbert Curve provides method of traversing image with side of length $2^{N}$. This can be utilized to define mapping from 2D to 1D and inverse. This mapping have one useful property. The same regions of an image are mapped to similar segments of line, regardless what the degree of this curve is. For graphical demonstration see lines above each curve in figure \ref{fig:hilbert}.

\subsection{ReNet modification}

We can utilize mapping defined by Hilbert Curve to reduce dimensionality of image. Then we can reduce number of recurrent neural networks in each ReNet layer. Instead of 4 RNN swapping image columns and rows we can introduce 2 RNN swaping image converted to 1D sequence with Hilbert Curve.

We can define image after mapping to 1D as: $x^{h} = \mathcal{H}(x)$. To keep reduction of activation dimensions after each modified ReNet layer we introduce patches as in original ReNet. In this case patch is defined as set of subsequent pixels: $P^{h} = \{ p_{i}^h \}$, $P^{h} \in \mathbb{R}^{w_p \textrm{x} c}$. As in original ReNet pixels from same patch are one input of recurrent neural network. If patch size is 1, then RNN in modified ReNet layer works as bidirectional recurrent neural network. Activations of whole layers are concatenation of 2 RNN activations:

\begin{gather}
	v_{i}^{F} = f_{FWD}(v_{i-1}^{F}, p_{i}^{h}) \\
	v_{i}^{R} = f_{REV}(v_{i+1}^{F}, p_{i}^{h})
\end{gather}

To further speedup the computations, subsequent modified ReNet layers use activation provided by previous layer. No mapping or inverse mapping is needed.

By using Hilbert Curve we mean to hold some spatial information while reducing image size. By aplaying image size reduction introduced by patches, segments of activations sequence still correspond to fragments of image. This should reduce impact of reducing dimensionality of the data.

\section{EXPERIMENT SETUP}

\subsection{Datasets}

\begin{table}
\centering
\caption{Datasets used for experiments}
\label{tab:dataset}
\begin{tabular}{ |c|c|c|c|c|c| } 
 \hline
 dataset & \#images & \#classes & width & height & source \\ 
 \hline
 \makecell{Chest X-Ray\\ Images (Pneumonia)} & 5863 & 2 & \textnormal{>}1500 & \textnormal{>}1000 & \cite{xray-dataset}\\ 
 \hline
 \makecell{Flowers Recognition} & 4242 & 5 & 320 & 240 & \cite{flowers-dataset} \\ 
 \hline
 \makecell{Fashion MNIST} & 70000 & 10 & 28 & 28 & \cite{fashion-dataset} \\ 
 \hline
 \makecell{Natural Images} & 6899 & 8 & \textnormal{>}200 & \textnormal{>}50 & \cite{natural-img-dataset} \\ 
 \hline
\end{tabular}
\end{table}

Comparison of used datasets is given in table \ref{tab:dataset}. Chest X-Ray, Flowers Recognition and Natural Images datasets were downloaded with usage of kaggle public API in version respectively 2, 2 and 1. Fashion MNIST dataset was downloaded using keras library. In case of x-ray dataset conversion to greyscale was applied. All images were resized to (64,64) or (32,32) size, due to issues with to low GPU memory when using ReNet networks on larger images. This limits the field of application of ReNet to small images only.

\subsection{Used tools and hardware}

Experiments were performed using Google Colab, Google Clound Platform with usage of AI Platform and local PC with IntelCore i7 8700 processor and graphic card GTX 1070Ti. Code was implemented using keras library with tensorflow backend. Scikit-learn and numpy were also used.

\subsection{Hyperparameters finetuning}

To choose best model structure and hyperparameters modified version of Grid Search was applied. In normal setup Grid Search is complete search of defined hyperparameters space. Instead of complete search greedy search was introduced. Each value of hyperparameter was chosen separately. Order of choosing hyperparams was as follows: learning rates and regularization terms, then number of ReNet or convolutional layers with dropout probability, number of neurons in ReNet or convolutional layers and finally number of neurons in fully conected layers with dropout probability. This is a naive approach because we assume independent influence of controlled factors on process of learning. 

\section{RESULTS}

\section{CONCLUSIONS}


Code used for perforing tests is avaliable online in \cite{repo}.


\bibliographystyle{unsrt}
\bibliography{refs}

\end{document}
